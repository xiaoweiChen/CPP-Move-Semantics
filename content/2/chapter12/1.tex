在泛型代码中,我们经常计算一个值,然后返回给调用者。问题是,我们如何完美地返回值,但仍然保留类型和值类别?换句话说:应该如何声明以下函数的返回类型:\par

\begin{lstlisting}[caption={}]
template<typename T>
??? callFoo(T&& arg)
{
	return foo(std::forward<T>(arg));
}
\end{lstlisting}

这个函数中,调用了一个名为foo()的函数,形参是完全转发的arg。我们不知道这种类型的foo()返回什么;它可能是一个临时值(prvalue)、一个lvalue引用或一个rvalue引用。返回类型可以是const或非const。\par

那么,如何完美地将foo()的返回值返回给callFoo()的调用者呢?有几个种方式是不起作用的:\par

\begin{itemize}
	\item 返回类型auto将删除foo()返回类型的引用。例如,如果提供对容器元素的访问权限(将foo()视为at()成员函数或vector的索引操作符),callFoo()将不再提供对该元素的访问权限。此外,我们可能会创建一个不必要的副本(如果没有优化掉的话)。
	\item 任何作为引用的返回类型(auto\&, const auto\&,和auto\& \&)将返回对局部对象的引用,如果foo()按值返回一个临时对象。幸运的是,编译器在检测到此类bug时可以发出警告。
\end{itemize}

也就是说,我们需要一种方式来表示:\par

\begin{itemize}
	\item 如果我们有一个值,则按值返回
	\item 如果我们得到/有一个引用,则按引用返回
\end{itemize}

但仍然保留返回的类型和值类别。\par

C++14为此引入了一个新的占位符类型:decltype(auto)。\par

\begin{lstlisting}[caption={}]
template<typename T>
decltype(auto) callFoo(T&& arg) // since C++14
{
	return foo(std::forward<T>(arg));
}
\end{lstlisting}

有了这个声明,callFoo()按值返回,如果foo()按值返回,如果foo()按引用返回,callFoo()按引用返回。在所有情况下,类型和值类别都保留。\par











































