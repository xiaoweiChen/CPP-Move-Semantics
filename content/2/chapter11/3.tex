再次注意,通用引用是将引用绑定到任何类型和值类别的任何对象,并且仍然保留其值类别和是否为\textit{const}的唯一方法。这也适用于用auto\&\&声明的通用引用。\par

\hspace*{\fill} \par %插入空行
\textbf{11.3.1 通用引用和基于范围的for循环}

使用基于范围的for循环时,使用auto\&\&声明的非转发通用引用起着重要的作用。\par

\hspace*{\fill} \par %插入空行
\textbf{基于范围的循环规范}

在C++标准中,基于范围的for循环可以指定使用普通for循环遍历范围内的元素。\par

类似这样的调用:\par

\begin{lstlisting}[caption={}]
std::vector<std::string> coll;
...
for (const auto& s : coll) {
	...
}
\end{lstlisting}

相当于:\par

\begin{lstlisting}[caption={}]
std::vector<std::string> coll;
...
auto&& range = coll; // initialize a universal reference
auto pos = range.begin(); // to use the given range coll here
auto end = range.end(); // and here
for ( ; pos != end; ++pos ) {
	const auto& s = *pos;
	...
}
\end{lstlisting}

将range声明为一个通用引用,希望能够将其绑定到每个range,所以可以使用它两次(一次是开始,一次是结束)没有创建副本或丢失信息。\par

循环应该为:\par

\begin{itemize}
	\item 非\textit{const} lvalue:
	\begin{lstlisting}[caption={}]
	std::vector<int> coll;
	...
	for (int& i : coll) {
		i *= 2;
	}
	\end{lstlisting}
	\item \textit{const} lvalue:
	\begin{lstlisting}[caption={}]
	const std::vector<int> coll{0, 8, 15};
	...
	for (int i : coll) {
		...
	}
	\end{lstlisting}
	\item prvalue:
	\begin{lstlisting}[caption={}]
	for (int i : std::vector<int>{0, 8, 15}) {
		...
	}
	\end{lstlisting}
\end{itemize}

注意,对于这些情况,没有其他方法声明range:\par

\begin{itemize}
	\item 使用auto,将创建range的副本(这需要花费时间并禁用对元素的修改)。
	\item 使用auto\&,可以用临时的prvalue禁用range的初始化。
	\item 使用const auto\&,将失去所遍历range的非常量性。
\end{itemize}

请注意,现在指定的基于范围的for循环有一个问题。代码如以下:\par

\begin{lstlisting}[caption={}]
	std::vector<std::string> createStrings();
	...
	for (char c : createStrings().at(0)) { // fatal runtime error
		...
	}
\end{lstlisting}

成为:\par

\begin{lstlisting}[caption={}]
	std::vector<std::string> createStrings();
	...
	auto&& range = createStrings().at(0); // OOPS: universal reference to reference
	auto pos = range.begin(); // return value of createStrings() destroyed here
	auto end = range.end();
	for ( ; pos != end; ++pos ) {
		char c = *pos;
		...
	}
\end{lstlisting}

所有有效引用都延长了所绑定值的生命周期,这也适用于rvalue引用。但是,没有绑定到\textit{createString()}的返回值(这样可以正常工作);而是绑定到引用,该引用指向由\textit{at()}返回的\textit{createStrings()}的返回类型,扩展了引用的生命周期。因此,该循环将遍历已经销毁的字符串。\par

\hspace*{\fill} \par %插入空行
\textbf{使用基于范围的for循环}

即使在调用基于范围的for循环时,通用引用也有意义。\par

要在迭代时修改元素,必须使用非\textit{const}引用。考虑函数模板,将传递值赋给传递集合中的所有元素:\par

\begin{lstlisting}[caption={}]
template<typename Coll, typename T>
void assign(Coll& coll, const T& value) {
	for (auto& elem : coll) {
		elem = value;
	}
}
\end{lstlisting}

看起来适用于所有容器类型和元素类型(其中支持赋值):\par

\begin{lstlisting}[caption={}]
std::vector<int> coll1{0, 8, 15};
...
assign(coll1, 42); // OK

std::vector<std::string> coll2{"hello", "world"};
...
assign(coll2, "ok"); // OK
\end{lstlisting}

然而,有种情况行不通:\par

\begin{lstlisting}[caption={}]
std::vector<bool> collB{false, true, false};
...
assign(collB, true); // ERROR: cannot bind non-const lvalue reference to an rvalue
\end{lstlisting}

发生了什么事?看一下基于范围的for循环展开的代码:\par

\begin{lstlisting}[caption={}]
std::vector<bool> coll{false, true, false};
...
{
	auto&& range = coll; // OK: universal reference to reference
	auto pos = range.begin(); // OK
	auto end = range.end(); // OK
	for ( ; pos != end; ++pos ) { // OK
		auto& elem = *pos; // ERROR: cannot bind non-const lvalue reference to an rvalue
		elem = elem + elem;
	}
}
\end{lstlisting}

问题是std::vector<bool>中的元素不是bool类型的对象,而是单个比特位。实现方法是:对于std::vector<bool>,元素引用的类型不是元素类型的引用。std::vector<bool>的实现是主模板std::vector<T>实现的偏特化,其中元素的引用是代理类的对象,可以像引用一样使用:\par

\begin{lstlisting}[caption={}]
namespace std {
	template< ... >
	class vector<bool, ... > {
		public:
		...
		class reference {
			...
		};
		...
	};
}
\end{lstlisting}

当对迭代器解引用时,返回std::vector<bool>::reference的值。因此,在基于范围的for循环的扩展代码中的语句为\par

\begin{lstlisting}[caption={}]
auto& elem = *pos;
\end{lstlisting}

尝试将非\textit{const} lvalue引用绑定到临时对象(prvalue),这是不允许的。\par

然而,这个问题有一个解决方案:调用基于范围的for循环时使用通用引用:\par

\begin{lstlisting}[caption={}]
template<typename Coll, typename T>
void assign(Coll& coll, const T& value) {
	for (auto&& elem : coll) { // note: universal reference support proxy element types
		elem = value;
	}
}
\end{lstlisting}

因为通用引用可以绑定到任何对象(甚至是prvalue),所以vector<bool>的代码现在可以编译:\par

\begin{lstlisting}[caption={}]
std::vector<bool> collB{false, true, false};
...
assign(collB, true); // OK (universal reference used to bind to an element)
\end{lstlisting}

因此,找到了使用非转发通用引用的另一个原因:绑定到没有作为引用实现的引用类型。或者说:允许绑定作为代理类型提供的非\textit{const}对象来进行操作。\par

































