C++标准库提供了两种词汇表类型,用于使用值语义处理一个或多个值(对象自动地和整体地保存和复制它们的值)。\par

原则上,都提供了移动语义。然而,其中有一些需要专门讨论和说明。\par

\hspace*{\fill} \par %插入空行
\textbf{15.3.1 pair的移动语义}

std::pair<>是很好的例子,展示了移动语义的好处和复杂性。原则上,只有一个具有两个成员的通用数据结构(在命名空间std中定义):\par

\begin{lstlisting}[caption={}]
template<typename T1, typename T2>
struct pair {
	T1 first;
	T2 second;
	...
};
\end{lstlisting}

为了支持移动语义(以及其他一些棘手的案例,如引用成员),有以下声明(这里,使用了C++14版本,并做了一些利于可读性的工作):\par

\begin{lstlisting}[caption={}]
template<typename T1, typename T2>
struct pair {
	// types of each member:
	using first_type = T1; // same as: typedef T1 first_type
	using second_type = T2;
	// the members:
	T1 first;
	T2 second;
	
	// constructors:
	constexpr pair();
	constexpr pair(const T1& x, const T2& y);
	template<typename U, typename V> constexpr pair(U&& x, V&& y);
	pair(const pair&) = default;
	pair(pair&&) = default;
	template<typename U, typename V> constexpr pair(const pair<U, V>& p);
	template<typename U, typename V> constexpr pair(pair<U, V>&& p);
	template<typename... Args1, typename... Args2>
	pair(piecewise_construct_t, tuple<Args1...> first_args,
	tuple<Args2...> second_args);
	
	// assignments:
	pair& operator=(const pair& p);
	pair& operator=(pair&& p) noexcept( ... );
	template<typename U, typename V> pair& operator=(const pair<U, V>& p);
	template<typename U, typename V> pair& operator=(pair<U, V>&& p);
	
	// other:
	void swap(pair& p) noexcept( ... );
};
\end{lstlisting}

该类支持移动语义。有一个默认的移动构造函数和一个已实现的移动赋值操作符(如果两种成员类型都保证不抛出异常,则对应的noexcept条件是不抛出异常):\par

\begin{lstlisting}[caption={}]
template<typename T1, typename T2>
struct pair {
	...
	pair(pair&&) = default;
	...
	pair& operator=(pair&& p) noexcept( ... );
	...
};
\end{lstlisting}

因此,代码如下:\par

\begin{lstlisting}[caption={}]
std::pair<std::string, std::string> p1{"some value", "some other value"};
auto p2{p1};
auto p3{std::move(p1)};
std::cout << "p1: " << p1.first << '/' << p1.second << '\n';
std::cout << "p2: " << p2.first << '/' << p2.second << '\n';
std::cout << "p3: " << p3.first << '/' << p3.second << '\n';
\end{lstlisting}

有如下输出:\par

\begin{tcolorbox}[colback=white,colframe=black]
p1: / \\
p2: some value/some other value \\
p3: some value/some other value
\end{tcolorbox}

但是,std::pair<>也可以通过处理通用/转发引用支持完美转发:\par

\begin{lstlisting}[caption={}]
template<typename T1, typename T2>
struct pair {
	...
	template<typename U, typename V> constexpr pair(U&& x, V&& y);
	...
};
\end{lstlisting}

因此,可以在初始化pair时使用移动语义。例如:\par

\begin{lstlisting}[caption={}]
int val = 42;
std::string s1{"value of s1"};
std::pair<std::string, std::string> p4{std::to_string(val), std::move(s1)};

std::cout << "s1: " << s1 << '\n';
std::cout << "p4: " << p4.first << '/' << p4.second << '\n';
\end{lstlisting}

有如下输出:\par

\begin{tcolorbox}[colback=white,colframe=black]
s1: \\
p4: 42/value of s1
\end{tcolorbox}

相应的成员模板实现为期望的通用/转发引用:\par

\begin{lstlisting}[caption={}]
template<typename U, typename V>
constexpr pair::pair(U&& x, V&& y)
: first(std::forward<U>(x)), second(std::forward<V>(y)) {
}
\end{lstlisting}

注意,通用引用/转发引用还意味着,只要定义了相应的类型转换,就可以创建和赋值不同的pair。例如:\par

\begin{lstlisting}[caption={}]
std::pair<const char*, std::string> p5{"answer", "is 42"};
auto p6{std::move(p5)};

std::cout << "p5: " << p5.first << '/' << p5.second << '\n';
std::cout << "p6: " << p6.first << '/' << p6.second << '\n';
\end{lstlisting}

有如下输出:\par

\begin{tcolorbox}[colback=white,colframe=black]
p5: answer/ \\
p6: answer/is 42
\end{tcolorbox}

初始化\textit{p6}时,将\textit{p5}的第一个成员(声明为const char*)转换为std::string,而在使用\textit{p5}的第二个成员初始化\textit{p6}的第二个成员时使用移动语义。\par

最后,注意std::pair<>支持具有引用类型的成员。在这种情况下,当为这些成员使用\textit{std::move()}时将应用特殊规则。参见basics/members.cpp获取完整的示例。\par

\hspace*{\fill} \par %插入空行
\textbf{std::make\_pair()}

std::pair<>附带了一个方便的函数模板std::make\_pair<>(),用于创建pair而不必指定成员的类型:\par

\begin{lstlisting}[caption={}]
auto p{std::make_pair(42, "hello")}; // creates std::pair<int, const char*>
\end{lstlisting}

std::make\_pair<>()是一个很好的例子,它演示了在rvalue和通用/转发引用中使用移动语义时必须考虑的另一件事。它的声明在不同的C++标准中有所不同:\par

\begin{itemize}
	\item 第一个C++标准是C++98中,make\_pair<>()是在命名空间std中使用引用来声明的,以避免不必要的复制:
	\begin{lstlisting}[caption={}]
	template<typename T1, typename T2>
	pair<T1,T2> make_pair (const T1& a, const T2& b)
	{
		return pair<T1,T2>(a,b);
	}
	\end{lstlisting}
	然而,当使用成对的字符串字面值或原始数组时,会导致了严重的问题。例如,当"hello"作为第二个实参传递时,对应形参\textit{b}的类型成为对const char数组(const char(\&)[6])的引用。因此,char类型[6]推导为T2类型,并用作第二个成员的类型。但是,不能使用数组初始化数组成员,因为不能复制数组。\par
	这种情况下,应该使用衰变的类型作为成员类型,这是按值传递参数时获得的类型(const char*表示字符串)。
	
	\item 因此,C++03中,函数定义改为使用按值调用:
	\begin{lstlisting}[caption={}]
	template<typename T1, typename T2>
	pair<T1,T2> make_pair (T1 a, T2 b)
	{
		return pair<T1,T2>(a,b);
	}
	\end{lstlisting}
	正如在问题解决方案的基本原理中看到的那样,“这似乎是对标准的建议小得多的更改,并且效率方面都被解决方案的优势抵消了。”
	
	\item C++11中,make\_pair()必须支持移动语义,这意味着参数必须成为通用/转发引用。同样,对于引用,参数的类型不会衰减。因此,定义变更如下:
	\begin{lstlisting}[caption={}]
	template<typename T1, typename T2>
	constexpr pair<typename decay<T1>::type, typename decay<T2>::type>
	make_pair (T1&& a, T2&& b)
	{
		return pair<typename decay<T1>::type,
		typename decay<T2>::type>(forward<T1>(a),
		forward<T2>(b));
	}
	\end{lstlisting}
	C++14中可以写成这样:
	\begin{lstlisting}[caption={}]
	template<typename T1, typename T2>
	constexpr pair<decay_t<T1>, decay_t<T2>>
	make_pair (T1&& a, T2&& b)
	{
		return pair<decay_t<T1>, decay_t<T2>>(forward<T1>(a), forward<T2>(b));
	}
	\end{lstlisting}
\end{itemize}

真正的实现更加复杂,因为C++11:为了支持\textit{std::ref()}和\textit{std::cref()},还使用了\textit{std::reference\_wrapper<>}。\par

C++标准库以类似的方式在许多地方完美地转发参数,通常还会结合使用std::decay<>。\par

\hspace*{\fill} \par %插入空行
\textbf{15.3.2 std::optional<>的移动语义}

std::optional<>是C++17中可用的值类型,通过“没有任何值”扩展包含所有可能的类型值。这避免了为具有此语义而标记该类型的特定值(例如,指针值0)。\par

optional对象也支持移动语义。如果将对象作为一个整体移动,状态将复制,所包含的对象(如果有的话)将移动。因此,一个已移动的对象仍然具有相同的状态,但任何值都变成未定义的。\par

但也可以将值移进或移出所包含的对象。例如:\par

\begin{lstlisting}[caption={}]
std::optional<std::string> os;
std::string s = "a very very very long string";
os = std::move(s); // OK, moves
std::string s2 = *os; // OK, copies
std::string s3 = std::move(*os); // OK, moves
\end{lstlisting}

注意,在最后一次调用之后,\textit{os}仍然有一个字符串值,但与通常的已移动对象一样,这个值未定义。因此,只要不对其值做任何假设,就可以使用它。甚至可以在那里赋一个新的字符串值。\par

还请注意,有些重载确保临时optional可移动。考虑一个返回可选字符串的函数:\par

\begin{lstlisting}[caption={}]
std::optional<std::string> func();
\end{lstlisting}

这种情况下,定义了移动值:\par

\begin{lstlisting}[caption={}]
std::string s4 = func().value(); // OK, moves
std::string s5 = *func(); // OK, moves
\end{lstlisting}

这种行为可以通过使用引用限定符,为相应的成员函数提供rvalue重载来实现:\par

\begin{lstlisting}[caption={}]
namespace std {
	template<typename T>
	class optional {
		...
		constexpr T& operator*() &;
		constexpr const T& operator*() const&;
		constexpr T&& operator*() &&;
		constexpr const T&& operator*() const&&;
		
		constexpr T& value() &;
		constexpr const T& value() const&;
		constexpr T&& value() &&;
		constexpr const T&& value() const&&;
	};
}
\end{lstlisting}

通过使用引用限定符,类可以在对rvalue(临时对象或标记为\textit{std::move()}的对象)调用操作时返回移动值。:\par

\begin{lstlisting}[caption={}]
std::vector<std::string> coll;
std::optional<std::string> optStr;
...
coll.push_back(std::move(optStr).value()); // OK, moves from member into coll
\end{lstlisting}

注意,std::optional<>是C++标准库中少数使用\textit{const} ravlue引用的类型。原因是std::optional<>是一个包装器类型,它希望确保操作正确,即使\textit{const}对象标记为\textit{std::move()},并且所包含的类型为const rvalue引用提供了特殊行为。\par

















