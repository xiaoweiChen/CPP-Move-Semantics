
\documentclass[11pt,a4paper,UTF8]{ctexart}

\usepackage[T1]{fontenc}
\usepackage[utf8]{inputenc}
\usepackage{authblk}

\usepackage{ctex} %导入中文包
\usepackage{tocvsec2}

\usepackage{tabularx}
\usepackage{booktabs} 
\usepackage{multirow}
\usepackage{bbding}
\usepackage{float}
\usepackage{xspace}

\usepackage{graphicx}
\usepackage{subfigure}

\usepackage{subfiles} %使用多文件方式进行

\usepackage{geometry} %设置页边距的包
\geometry{left=2.5cm,right=2cm,top=2.54cm,bottom=2.54cm} %设置书籍的页边距

\usepackage{hyperref}  %制作pdf的目录
\hypersetup{hidelinks, %去红框
	colorlinks=true,
	allcolors=black,
	pdfstartview=Fit,
	breaklinks=true
}

% 调整itemlist中的行间距
\usepackage{enumitem}
\setenumerate[1]{itemsep=0pt,partopsep=0pt,parsep=\parskip,topsep=5pt}
\setitemize[1]{itemsep=0pt,partopsep=0pt,parsep=\parskip,topsep=5pt}
\setdescription{itemsep=0pt,partopsep=0pt,parsep=\parskip,topsep=5pt}

% 超链接样式设置
\usepackage{hyperref}
\hypersetup{
	colorlinks=true,
	linkcolor=blue,
	filecolor=blue,
	urlcolor=blue,
	citecolor=cyan,
}

\usepackage{indentfirst}

\usepackage{listings}
\usepackage[usenames,dvipsnames,svgnames, x11names]{xcolor}

\usepackage{tcolorbox}

%展示代码
\definecolor{mygreen}{rgb}{0,0.6,0}
\definecolor{mygray}{rgb}{0.5,0.5,0.5}
\definecolor{mymauve}{rgb}{0.58,0,0.82}
\lstset{
	backgroundcolor=\color{blue!3!white}, 
	basicstyle = \footnotesize,       
	breakatwhitespace = false,        
	breaklines = true,                 
	captionpos = b,                    
	commentstyle = \color{mygreen}\bfseries,
	extendedchars = false,             
	frame =shadowbox, 
	framerule=0.5pt,
	keepspaces=true,
	keywordstyle=\color{blue}\bfseries, % keyword style
	language = C++,                     % the language of code
	otherkeywords={string}, 
	numbers=left, 
	numbersep=5pt,
	numberstyle=\tiny\color{mygray},
	rulecolor=\color{black},         
	showspaces=false,  
	showstringspaces=false, 
	showtabs=false,    
	stepnumber=1,         
	stringstyle=\color{mymauve},        % string literal style
	tabsize=2,          
	title=\lstname                      
}

\begin{document}
	%\maketitle
	
	\begin{center}
		\includegraphics[width=1.\textwidth]{cover}
		\newpage
		\huge
		\textbf{C++ Move Semantics} 
		\\[9pt]
		\normalsize
		The Complete Guide
		\\[10pt]
		\normalsize 
		作者: Nicolai M. Josuttis
		\\[8pt]
		\normalsize
		译者;陈晓伟
	\end{center}
	
	\hspace*{\fill} \\ %插入空行
	\noindent\textbf{本书概述}\ \par

	完整的介绍C++ Move语义。\par
	
	C++11添加的Move语义已经成为现代C++的标志,也使语言变得复杂,即使经验丰富的开发者仍在需要仔细处理Move语义的细节。因为这个原因,一些编程书籍甚至不推荐对非常简单的类使用Move语义。所以,详细的解释C++ Move语义就变得刻不容缓。\par
	
	本书会从基本原理开始来介绍Move语义,并会解释Move语义的所有细节,使每个开发者都可以正确地使用Move语义。\par
	
	本书项目始于2019年,对于完全支持C++和数据并行的需要大量的扩展,超出当时的SYCL 1.2.1标准。DPC++编译器需要支持这些扩展,包括对统一共享内存(USM)的支持、通过SYCL完成三级层次结构的子工作组、匿名Lambda和许多编程简化。\par
	
	你将学习到:\par
	
	\begin{itemize}
		\item Move语义的起因和术语
		\item 如何隐式地获益于Move语义
		\item 如何明确地获益于Move语义
		\item 会遇到的所有问题,以及如何处理它们
		\item 所有的结果都取决于你的编程风格
	\end{itemize}
	
	重点在于所描述的特性,需要在实践中进行应用。示例和背景信息,有助于理解和改进简单类,甚至泛型库和框架的代码。\par
	
	“我以为我理解了Move的语义,但我真的不懂!”我从你的书中学到了很多东西。”	-- Jonathan Boccara\par
	
	“这是我需要很久的书。” -- Rob Bernstein\par
	
	“有时候我觉得我对纠缠和量子隐形传态的理解,要比我对一些奇怪的C++ Move语义的理解要好。套用Feynman的话:如果你认为你理解了C++的Move语义,那你就不理解C++的Move语义。赶快阅读这本书吧。”	-- Victor Ciura\par
	
	
	\hspace*{\fill} \\ %插入空行
	\noindent\textbf{作者简介}\ \par
	Nicolai Josuttis (http://www.josuttis.com)在编程界很有名,因为他的发言和著作都很有权威,还是世界范围内畅销书的(共同)作者:\par
	
	\begin{itemize}
		\item 《The C++ Standard Library》
		\item 《C++ Templates》
		\item 《C++ Move Semantics》
		\item 《C++17》
		\item 《SOA in Practice》
	\end{itemize}
	
	同时也是一位富有创新精神的演讲者,曾在各种会议和活动中发言。还是独立的讲师,并且在C++标准化方面有20多年的经验。\par
	
	\hspace*{\fill} \\ %插入空行
	\noindent\textbf{本书相关}\ \par
	\begin{itemize}
		\item github翻译地址:\href{https://github.com/xiaoweiChen/CPP-Move-Semantics}{https://github.com/xiaoweiChen/CPP-Move-Semantics}
	\end{itemize}
	\newpage
	
	\tableofcontents
	\newpage
	
	%前言和关于本书
	\pagestyle{empty}
	\subfile{content/Preface.tex}
	\subfile{content/About_This_Book.tex}

	\setsecnumdepth{section}
	\section{Part I: 移动语义的基本特征}
	\subfile{content/1/Part-1.tex}
	\subsection{移动语义的力量}
	\subfile{content/1/chapter1/0.tex}
		\subsubsection{设计移动语义的动机}
		\subfile{content/1/chapter1/1.tex}
		\subsubsection{实现移动语义}
		\subfile{content/1/chapter1/2.tex}
		\subsubsection{复制是一种应急方式}
		\subfile{content/1/chapter1/3.tex}
		\subsubsection{const对象的移动语义}
		\subfile{content/1/chapter1/4.tex}
		\subsubsection{总结}
		\subfile{content/1/chapter1/5.tex}
	\subsection{Core Features of Move Semantics}
	%\subfile{content/chapter-2/2-0.tex}
		\subsubsection{Rvalue References}
		%\subfile{content/chapter-2/2-1.tex}
		\subsubsection{std::move()}
		%\subfile{content/chapter-2/2-2.tex}
		\subsubsection{Moved-From Objects}
		%\subfile{content/chapter-2/2-3.tex}
		\subsubsection{Overloading by Different References}
		%\subfile{content/chapter-2/2-4.tex}
		\subsubsection{Passing by Value}
		%\subfile{content/chapter-2/2-4.tex}
		\subsubsection{总结}
		%\subfile{content/chapter-2/2-10.tex}
	\subsection{Move Semantics in Classes}
	%\subfile{content/chapter-3/3-0.tex}
		\subsubsection{Move Semantics in Ordinary Classes}
		%\subfile{content/chapter-3/3-1.tex}
		\subsubsection{Implementing Special Copy/Move Member Functions}
		%\subfile{content/chapter-3/3-2.tex}
		\subsubsection{Rules for Special Member Functions}
		%\subfile{content/chapter-3/3-4.tex}
		\subsubsection{The Rule of Five or Three}
		%\subfile{content/chapter-3/3-5.tex}
		\subsubsection{总结}
		%\subfile{content/chapter-3/3-12.tex}
	\subsection{How to Benefit From Move Semantics}
	%\subfile{content/chapter-4/4-0.tex}
		\subsubsection{Avoid Objects with Names}
		%\subfile{content/chapter-4/4-1.tex}
		\subsubsection{Avoid Unnecessary std::move()}
		%\subfile{content/chapter-4/4-2.tex}
		\subsubsection{Initialize Members with Move Semantics}
		%\subfile{content/chapter-4/4-3.tex}
		\subsubsection{Move Semantics in Class Hierarchies}
		%\subfile{content/chapter-4/4-4.tex}
		\subsubsection{总结}
		%\subfile{content/chapter-4/4-8.tex}
	\subsection{Overloading on Reference Qualifiers}
	%\subfile{content/chapter-5/5-0.tex}
		\subsubsection{Return Type of Getters}
		%\subfile{content/chapter-5/5-1.tex}
		\subsubsection{Overloading on Qualifiers}
		%\subfile{content/chapter-5/5-2.tex}
		\subsubsection{When to Use Reference Qualifiers}
		%\subfile{content/chapter-5/5-3.tex}
		\subsubsection{总结}
		%\subfile{content/chapter-5/5-6.tex}
	\subsection{Moved-From States}
	%\subfile{content/chapter-6/6-0.tex}
		\subsubsection{Required and Guaranteed States of Moved-From Objects}
		%\subfile{content/chapter-6/6-1.tex}
		\subsubsection{Destructible and Assignable}
		%\subfile{content/chapter-6/6-2.tex}
		\subsubsection{Dealing with Broken Invariants}
		%\subfile{content/chapter-6/6-3.tex}
		\subsubsection{总结}
		%\subfile{content/chapter-6/6-6.tex}
	\subsection{Move Semantics and noexcept}
	%\subfile{content/chapter-7/7-0.tex}
		\subsubsection{Move Constructors with and without noexcept}
		%\subfile{content/chapter-7/7-1.tex}
		\subsubsection{Details of noexcept Declarations}
		%\subfile{content/chapter-7/7-2.tex}
		\subsubsection{noexcept Declarations in Class Hierarchies}
		%\subfile{content/chapter-7/7-2.tex}
		\subsubsection{When and Where to Use noexcept}
		%\subfile{content/chapter-7/7-2.tex}
		\subsubsection{总结}
		%\subfile{content/chapter-7/7-3.tex}
	\subsection{Value Categories}
	%\subfile{content/chapter-8/8-0.tex}
		\subsubsection{Value Categories}
		%\subfile{content/chapter-8/8-1.tex}
		\subsection{Special Rules for Value Categories}
		%\subsubsection{content/chapter-8/8-2.tex}
		\subsubsection{Impact of Value Categories When Binding References}
		%\subfile{content/chapter-8/8-3.tex}
		\subsubsection{When Lvalues become Rvalues}
		%\subfile{content/chapter-8/8-4.tex}
		\subsubsection{When Rvalues become Lvalues}
		%\subfile{content/chapter-8/8-4.tex}
		\subsubsection{Checking Value Categories with decltype}
		%\subfile{content/chapter-8/8-4.tex}
		\subsubsection{总结}
		%\subfile{content/chapter-8/8-5.tex}
		
	\section{Part II: Move Semantics in Generic Code}
	\subsection{Perfect Forwarding}
	%\subfile{content/chapter-9/9-0.tex}
		\subsubsection{Motivation for Perfect Forwarding}
		%\subfile{content/chapter-9/9-1.tex}
		\subsubsection{Implementing Perfect Forwarding}
		%\subfile{content/chapter-9/9-2.tex}
		\subsubsection{Rvalue References versus Universal References}
		%\subfile{content/chapter-9/9-3.tex}
		\subsubsection{Overload Resolution with Universal References}
		%\subfile{content/chapter-9/9-4.tex}
		\subsubsection{Perfect Forwarding in Lambdas}
		%\subfile{content/chapter-9/9-5.tex}
		\subsubsection{总结}
		%\subfile{content/chapter-9/9-6.tex}
	\subsection{Tricky Details of Perfect Forwarding}
	%\subfile{content/chapter-10/10-0.tex}
		\subsubsection{Universal References as Non-Forwarding References}
		%\subfile{content/chapter-10/10-1.tex}
		\subsubsection{Universal or Ordinary Rvalue Reference?}
		%\subfile{content/chapter-10/10-2.tex}
		\subsubsection{How the Standard Specifies Perfect Forwarding}
		%\subfile{content/chapter-10/10-3.tex}
		\subsubsection{Nasty Details of Perfect Forwarding}
		%\subfile{content/chapter-10/10-5.tex}
		\subsubsection{总结}
		%\subfile{content/chapter-10/10-6.tex}
	\subsection{Perfect Passing with auto\&\&}
	%\subfile{content/chapter-11/11-0.tex}
		\subsubsection{Default Perfect Passing}
		%\subfile{content/chapter-11/11-1.tex}
		\subsubsection{Universal References with auto\&\&}
		%\subfile{content/chapter-11/11-2.tex}
		\subsubsection{auto\&\& as Non-Forwarding Reference}
		%\subfile{content/chapter-11/11-3.tex}
		\subsubsection{Perfect Forwarding in Lambdas}
		%\subfile{content/chapter-11/11-4.tex}
		\subsubsection{Using auto\&\& in C++20 Function Declarations}
		%\subfile{content/chapter-11/11-5.tex}
		\subsubsection{总结}
		%\subfile{content/chapter-11/11-6.tex}
	\subsection{Perfect Returning with decltype(auto)}
	%\subfile{content/chapter-12/12-0.tex}
		\subsubsection{Perfect Returning}
		%\subfile{content/chapter-12/12-1.tex}
		\subsubsection{decltype(auto)}
		%\subfile{content/chapter-12/12-2.tex}
		\subsubsection{总结}
		%\subfile{content/chapter-12/12-8.tex}
		
	\section{Part III: Move Semantics in the C++ Standard Library}
	\subsection{Move-Only Types}
	%\subfile{content/chapter-13/13-0.tex}
		\subsubsection{Declaring and Using Move-Only Types}
		%\subfile{content/chapter-13/13-1.tex}
		\subsubsection{总结}
		%\subfile{content/chapter-13/13-10.tex}
	\subsection{Moving Algorithms and Iterators}
	%\subfile{content/chapter-14/14-0.tex}
		\subsubsection{Moving Algorithms}
		%\subfile{content/chapter-15/15-1.tex}
		\subsubsection{Removing Algorithms}
		%\subfile{content/chapter-15/15-2.tex}
		\subsubsection{Move Iterators}
		%\subfile{content/chapter-15/15-3.tex}
		\subsubsection{总结}
		%\subfile{content/chapter-15/15-5.tex}
	\subsection{Move Semantics in Types of the C++ Standard Library}
	%\subfile{content/chapter-15/0.tex}
		\subsubsection{Move Semantics for Strings}
		%\subfile{content/chapter-15/1.tex}
		\subsubsection{Move Semantics for Containers}
		%\subfile{content/chapter-15/2.tex}
		\subsubsection{Move Semantics for Vocabulary Types}
		%\subfile{content/chapter-15/3.tex}
		\subsubsection{Move Semantics for Smart Pointers}
		%\subfile{content/chapter-15/4.tex}
		\subsubsection{Move Semantics for IOStreams}
		%\subfile{content/chapter-15/5.tex}
		\subsubsection{Move Semantics for Multithreading}
		%\subfile{content/chapter-15/6.tex}
		\subsubsection{总结}
		%\subfile{content/chapter-15/6.tex}
\end{document}